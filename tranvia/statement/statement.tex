\documentclass{oci}
\usepackage[utf8]{inputenc}
\usepackage{lipsum}

\title{Suma de ejemplo}

\begin{document}
\begin{problemDescription}
  Consientes de la crisis climática y las catastróficas consecuencias que se avecinan si no reducimos
  las emisiones de carbono, la ciudad de Chuchunco ha decidido construir un nuevo sistema de transporte
  público masivo de cero emisiones.
  %
  Tras una larga evaluación, los ciudadanos de Chuchunco ha decidido que la mejor opción es construir un
  sistema de tren ligero o *tranvía*.

  En su primera fase, el proyecto contempla la construcción de una línea separada del tráfico con
  intersecciones a desnivel en puntos estratégicos.
  %
  La línea tendrá un largo $L$ y contará con $N$ estaciones.
  %
  Las estaciones serán numeradas de 1 a $N$ y la estación $i$ estará a una distancia $D_i$
  ($0 \leq D_i \leq L$) del inicio de la línea.
  %
  Para su apertura, también está considerada la compra de $T$ trenes articulados de piso bajo
  que serán numerados de 1 a $T$.

  La línea contará con un moderno sistema de Control de Trenes Basado en Comunicaciones (CBTC por
  sus siglas en inglés).
  %
  El sistema CBTC permitirá conocer la posición exacta de los trenes de forma de tener una gestión
  de tráfico eficiente y segura.
  %
  La ciudad de Chunchunco quiere aprovechar esta tecnología para instalar pantallas en todas las
  estaciones que muestren el tiempo estimado de llegada del próximo tren.

  Las pantallas obtendrán la información de un servidor central que mantendrá la posición de cada
  uno de los trenes.
  %
  La posición de un tren puede ser desconocida en caso de no estar en servicio o haber fallos en el
  sistema.
  %
  En caso contrario, la posición es representada con un entero $d$ ($0 \leq d\leq L$) correspondiente
  a la distancia desde el inicio de la línea.
  %
  El servidor actualizará periódicamente las posiciones de los trenes y basándose en estas, responderá
  cuál es el tiempo estimado de llegada del siguiente tren.
  %
  Específicamente, el servidor debe procesar dos tipos de consultas:
  \begin{enumerate}
    \item Dado un tren $t$ y una distancia $d$, actualizar la posición del tren $t$ para que sea $d$.
    \item Dada una estación $e$ calcular el tiempo de llegada del siguiente tren.
  \end{enumerate}
\end{problemDescription}

\begin{inputDescription}
  La entrada consiste en una sola línea con dos enteros $a$ y $b$ ($-2\cdot10^9 \leq a, b \leq 2\cdot10^9$).
\end{inputDescription}

\begin{outputDescription}
  La salida debe contener un único entero correspondiente a la suma de $a$ y $b$.
\end{outputDescription}

\begin{scoreDescription}
  \subtask{50} Se probarán varios casos de prueba donde $-10^9\leq a, b \leq 10^9$.
  \subtask{50} Se probarán varios casos de prueba sin restricciones adicionales.
\end{scoreDescription}

\begin{sampleDescription}
\sampleIO{sample-1}
\sampleIO{sample-2}
\end{sampleDescription}

\end{document}
