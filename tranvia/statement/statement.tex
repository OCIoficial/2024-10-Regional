\documentclass{oci}
\usepackage[utf8]{inputenc}
\usepackage{lipsum}

\title{Suma de ejemplo}

\begin{document}
\begin{problemDescription}
  Consientes de la crisis climática, la ciudad de Chuchunco
  han decidido construir un nuevo sistema de transporte
  público sustentable.
  %
  Tras una larga evaluación, el consejo ciudadano
  ha decidido que la mejor opción es construir un
  sistema de tren ligero.

  En su primera fase, el proyecto contempla la
  construcción de una línea separada del tráfico, con
  intersecciones a desnivel en puntos estratégicos.
  %
  La línea tendrá un largo $L$ y contará con $N$
  estaciones numeradas de oeste a este entre 1 y $N$.
  %
  La estación $i$ estará a una distancia $D_i$
  ($0 \leq D_i \leq L$) desde el extremo oeste
  de la línea.
  %
  La primera estación se construirá en el extremo
  oeste ($D_1 = 0$) y la última estación
  en el extremo este ($D_N = L$).
  %
  Para su apertura, también está considerada la
  compra de $M$ trenes articulados de piso bajo
  de última generación.
  %
  Estos trenes serán numerados de 1 a $M$.

  La línea contará con un moderno sistema de
  Control de Trenes Basado en Comunicaciones
  (CBTC por sus siglas en inglés).
  %
  El sistema CBTC permitirá conocer la posición
  exacta de los trenes de forma de tener una
  gestión de tráfico eficiente y segura.
  %
  Aprovechando esta tecnología, el consejo de
  transporte ha decidido instalar pantallas en todas las
  estaciones que muestren el tiempo estimado de
  llegada del próximo tren.

  Las pantallas obtendrán la información de un
  servidor central que mantendrá el estado
  de cada uno de los trenes.
  %
  En condiciones normales, el estado de un tren
  se representa con un par $(p, c)$ indicando
  su posición y dirección respectivamente.
  %
  El valor $p$ es un entero correspondiente a la
  distancia actual del tren desde el extremo oeste
  de la línea.
  %
  El valor $c$ es un carácter que será \texttt{E}
  si el tren se dirige hacia el este u \texttt{O}
  si se dirige hacia el oeste.
  %
  El estado de un tren también puede ser desconocido
  en caso de no estar en servicio o haber fallos
  de conexión.

  En un inicio, el estado de todos los trenes es
  desconocido.
  %
  A medida que estos entran en funcionamiento, el
  servidor va recibiendo eventos indicando el
  último estado conocido de los trenes.
  %
  Basándose en el último estado conocido, el servidor
  debe responder el tiempo estimado de llegada del
  próximo tren asumiendo que los trenes se seguirán
  moviendo en la misma dirección a una velocidad
  constante $V$.
  %
  Específicamente, el servidor deber procesar dos tipos
  de eventos:
  \begin{enumerate}
    \item \texttt{Estado j p c}: Indica que el último
    estado conocido del tren $j$ es $(p, c)$.
    %
    Si $p$ es $-1$ el estado del tren es desconocido.
    %
    En este caso el valor $c$ es irrelevante.
    \item \texttt{Tiempo i c}: Dada una estación $i$,
    el servidor debe responder el tiempo estimado de
    llegada del próximo tren en dirección $c$.
  \end{enumerate}
\end{problemDescription}

\begin{inputDescription}
  La entrada consiste en una sola línea con dos enteros $a$ y $b$ ($-2\cdot10^9 \leq a, b \leq 2\cdot10^9$).
\end{inputDescription}

\begin{outputDescription}
  La salida debe contener un único entero correspondiente a la suma de $a$ y $b$.
\end{outputDescription}

\begin{scoreDescription}
  \subtask{50} Se probarán varios casos de prueba donde $-10^9\leq a, b \leq 10^9$.
  \subtask{50} Se probarán varios casos de prueba sin restricciones adicionales.
\end{scoreDescription}

\begin{sampleDescription}
\sampleIO{sample-1}
\sampleIO{sample-2}
\end{sampleDescription}

\end{document}
