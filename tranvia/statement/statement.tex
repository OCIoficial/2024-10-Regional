\documentclass{oci}
\usepackage[utf8]{inputenc}
\usepackage{lipsum}

\title{Tranvía en Chuchunco}

\begin{document}
\begin{problemDescription}
  Consientes de la crisis climática, la ciudad de Chuchunco
  han decidido construir un nuevo sistema de transporte
  público sustentable.
  %
  Tras una larga evaluación, el consejo ciudadano de
  transporte ha decidido que la mejor opción es construir un
  \emph{tranvía}.

  En su primera fase, el proyecto contempla la
  construcción de una línea separada del tráfico, con
  intersecciones a desnivel en puntos estratégicos.
  %
  En esta primera fase también se espera que la línea
  opere en solo una dirección siguiendo el flujo del
  tráfico a la hora punta.

  La línea tendrá un largo $L$ y contará con $N$
  estaciones numeradas de 1 a $N$.
  %
  La estación $i$ estará a una distancia $d_i$
  ($0 \leq d_i \leq L$) desde el inicio de la línea.
  %
  Para su apertura, también está considerada la
  compra de $M$ trenes articulados de piso bajo
  de última generación.
  %
  Estos trenes serán numerados de 1 a $M$.

  La línea contará con un moderno sistema de
  Control de Trenes Basado en Comunicaciones
  (CBTC por sus siglas en inglés).
  %
  El sistema CBTC permitirá conocer la posición
  exacta de los trenes de forma de tener una
  gestión de tráfico eficiente y segura.
  %
  Aprovechando esta tecnología, el consejo de
  transporte ha decidido instalar pantallas en las
  estaciones que muestren el tiempo estimado de
  llegada del próximo tren.

  Las pantallas obtendrán la información de un
  servidor central que mantendrá el estado
  de cada uno de los trenes.
  %
  En condiciones normales, el estado de un tren
  se representa con un entero $p$ indicando
  su posición desde el inicio de la línea.
  %
  El estado de un tren también puede ser desconocido
  en caso de no estar en servicio o haber fallos
  de conexión.

  En un inicio, el estado de todos los trenes es
  desconocido.
  %
  A medida que estos entran en funcionamiento, el
  servidor va recibiendo eventos indicando el
  último estado conocido de los trenes.
  %
  Basándose en el último estado conocido, el servidor
  debe responder el tiempo estimado de llegada del
  próximo tren asumiendo que los trenes se seguirán
  moviendo a una velocidad constante unitaria.
  %
  Específicamente, para una estación $i$ a distancia $d_i$,
  decimos que el próximo tren corresponde al tren más cercano
  cuya posición actual es \textbf{menor o igual} que $d_i$.
  %
  Dada la posición $p$ del próximo tren, el tiempo estimado
  de llegada se calcula como $(d_i - p)$.

  Concretamente, el servidor deber responder a dos tipos
  de eventos:
  \begin{itemize}
    \item \textbf{Tipo 1}: Dada dos enteros $j$ y $p$, este evento
    indica que la última posición conocida del tren $j$ es $p$.
    %
    Si $p$ es igual a $-1$ entonces la posición del tren es desconocida
    y este ya no debe ser considerado al responder los eventos de tipo 2.
    \item \textbf{Tipo 2}: Dada una estación $i$, el servidor debe
    responder el tiempo estimado de llegada del próximo tren o determinar
    que hay ningún tren que se aproxima.
  \end{itemize}

  Notar que debido a fluctuaciones en el sistema, no hay garantías
  en como evolucionan las posiciones de los trenes.
  %
  Estos pueden retroceder o desaparecer del sistema en cualquier momento.
  %
  El servidor siempre responderá a los eventos de tipo 2 usando
  la posición actual de los trenes.

  El consejo de transporte tiene poca experiencia en programación.
  ?`Podrías ayudarlos implementando el servidor?
\end{problemDescription}

\begin{inputDescription}
  La primera línea de la entrada contiene 5 enteros $L$, $N$, $M$ y $E$:
  \begin{itemize}
    \item $L$ es el largo de la línea ($0 < L \leq 10^9$)
    \item $N$ es la cantidad de estaciones ($0 < N \leq 10^5$)
    \item $M$ es la cantidad de trenes ($0 < M \leq 10^5$)
    \item $E$ es el número de eventos que debe procesar el servidor ($0 < E \leq 4\cdot 10^5$)
  \end{itemize}
  %
  %
  La segunda línea contiene $N$ enteros en orden creciente, todos entre 0 y $L$,
  correspondientes a las distancias donde se ubican cada una de las estaciones.
  %
  Se garantiza que todas las estaciones estarán en una posición distinta.

  Luego siguen $E$ líneas que describen cada uno de los eventos.
  %
  Cada línea comienza con un entero $t$ ($t=1$ o $t=2$) indicando el tipo de evento:
  %
  \begin{itemize}
  \item Si $t=1$ siguen dos enteros $j$ ($1\leq j \leq M$) y $p$ ($-1\leq p \leq L$),
  indicando que la última posición conocida del tren $j$ es $p$.
  %
  Si $p$ es igual a -1 la posición del tren es desconocida.
  %
  En caso contrario, $p$ indica la nueva posición del tren desde el inicio
  de la línea.
  \item Si $t=2$ sigue un entero $i$ indicando que el servidor debe responder el tiempo
  estimado de llegada del próximo tren para la estación $i$.
  \end{itemize}

  % Se garantiza que en ningún momento dos trenes estarán en la misma posición.
\end{inputDescription}

\begin{outputDescription}
  La salida debe contener la respuesta a cada evento de tipo 2.
  %
  Cada respuesta debe aparecer en una nueva línea en el orden en
  que los eventos aparecen en la entrada.
  %
  La respuesta debe ser un entero correspondiente al tiempo estimado de llegada
  del próximo tren o -1 en caso de no haber un próximo tren.
\end{outputDescription}

\begin{scoreDescription}
  \subtask{30} Se probarán varios casos de prueba donde $M \leq 10^3$ y $E \leq 10^3$.
  \subtask{70} Se probarán varios casos de prueba sin restricciones adicionales.
\end{scoreDescription}

\begin{sampleDescription}
\sampleIO{sample-1}
\end{sampleDescription}

\end{document}
