\documentclass{oci}
\usepackage[utf8]{inputenc}
\usepackage{lipsum}

\title{Apuesta}

\begin{document}
\begin{problemDescription}
Fernanda y tú decidieron hacer una apuesta de alto riesgo: Fernanda lanza una moneda. Si cae sello, pasarán a la Industria Chilena de Papas Caseras (ICPC) a comerse una porción de papas, de lo contario, se quedarán de brazos cruzados todo el día.
Lamentablemente, la moneda cayó cara, por lo que el destino decidió que se perderían de esta increíble oportunidad. Pero Fernanda, no contenta con el resultado, decidió hacer otra apuesta.

	Fernanda lanza $n$ monedas en una fila. Si en la mayoría estricta de segmentos\footnote{Un segmento es un arreglo contiguo de monedas. Por ejemplo, todas las monedas desde la tercera hasta la quinta.} de la fila, la mayoría de las monedas caen sello, irán a comer papas de todas formas.

Una vez que lanzó las monedas, Fernanda se puso a contar la cantidad de segmentos exitosos y totales utilizando sus conocimentos avanzados de combinatoria.

Sin embargo, le está tomando demasiado tiempo y la ICPC cierra en menos de 4 horas, por lo que decides sacas tu computador y te pones a programar una solución.
\end{problemDescription}

\begin{inputDescription}
	La primera linea contiene un único entero $n$ $(1 \leq n \leq 10^6)$, la cantidad de monedas.

La segunda línea contiene $n$ enteros $m_i$, el resultado de la $i$-ésima moneda de la fila. $m_i$ es $-1$ si la moneda cayó sello, y $1$ en caso de haber caido cara.
\end{inputDescription}

\begin{outputDescription}
	La salida debe contener dos enteros $a$, $b$, que corresponden respectivamente a la cantidad de segmentos donde la mayoría estricta de monedas es sello, y la cantidad total de segmentos.
\end{outputDescription}

\begin{scoreDescription}
	\subtask{10} Se probarán varios casos de prueba donde $n \leq 4$
	\subtask{20} Se probarán varios casos de prueba donde $n \leq 100$ (valor referencial, $O(n^3)$ debería funcionar)
  	\subtask{30} Se probarán varios casos de prueba donde $n \leq 10^4$ (valor referencial, $O(n^2)$ debería funcionar)
	\subtask{40} Se probarán varios casos de prueba sin restricciones adicionales. ($10^6$ como valor referencial, $O(n log n)$ debería funcionar)
\end{scoreDescription}

\begin{sampleDescription}
\sampleIO{sample-1}
\sampleIO{sample-2}
\end{sampleDescription}

\end{document}
