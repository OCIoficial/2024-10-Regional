\documentclass{oci}
\usepackage[utf8]{inputenc}
\usepackage{lipsum}

\title{Ricardo y su nóctulo}

\begin{document}
\begin{problemDescription}
Ricardo no tiene ni un perro ni un gato.
%
Ricardo tiene una mascota muy exótica:
un nóctulo.
%
Para los que no sepan, un nóctulo es un murciélago chico que
vive en el desván.
%
Ricardo y su nóctulo casi nunca se ven.
%
El problema es que ambos tienen horarios de trabajo
muy rígidos y por lo tanto es difícil que sus
horarios libres coincidan.

Dependiendo del día de la semana, Ricardo trabaja de día o de noche.
%
Su horario se describe como una cadena de $7$ caracteres en dónde cada
carácter representa su horario de trabajo ese día de la semana:
\texttt{'D'} significa que trabaja de día y \texttt{'N'} que trabaja de noche.
%
Por ejemplo, la cadena \texttt{"DDDDNNN"} representa que
Ricardo trabaja de día los lunes, martes, miércoles y jueves,
y que trabaja de noche los viernes, sábados y domingos.

Los nóctulos no funcionan de la misma manera, pues
ellos tienen una semana de $5$ días: noctulunes,
noctumartes, noctumiércoles, noctujueves y noctuviernes,
y después del noctuviernes vuelve a ser noctulunes.
%
Por lo tanto, para representar el horario del nóctulo se usa
una cadena de $5$ caracteres.

Ricardo y su nóctulo tienen muchas ganas de verse, pero no
saben cuándo ambos van a estar libres al mismo tiempo.
%
Si ambos trabajan de día, se pueden ver en la noche, si ambos
trabajan de noche se pueden ver en el día, pero si no trabajan
al mismo tiempo, no se pueden ver ese día.
%
Hoy justo es lunes y noctulunes al mismo tiempo,
%
¡ayúdalos determinando cuántos días deben esperar para poderse ver!
%
Si nunca se van a ver, tienes que darles las malas noticias imprimiendo
\texttt{"No nos vemos nunca"}.
\end{problemDescription}

\begin{inputDescription}
La entrada consiste de dos líneas.

La primera línea contiene la cadena que representa el horario de Ricardo.
%
La cadena será de largo 7 y estará formada solo de los caracteres \texttt{'D'}
y \texttt{'N'}.

La segunda línea contiene la cadena que representa el horario del nóctulo de
Ricardo.
%
La cadena será de largo 5 y estará formada solo de los caracteres \texttt{'D'}
y \texttt{'N'}.
\end{inputDescription}

\begin{outputDescription}
La salida debe contener un único entero que represente en cuántos días más Ricardo y su nóctulo
podrán verse.
%
Es decir, cuántos días deben esperar para que ambos trabajen al mismo tiempo
considerando que hoy es lunes y noctulunes al mismo tiempo.
%
Si sus horarios nunca coinciden y por lo tanto no pueden verse nunca, debes
imprimir \texttt{"No nos vemos nunca"}.
\end{outputDescription}

\begin{scoreDescription}
  \subtask{30} Se probarán varios casos de prueba en los que se garantiza que van a poder verse en los primeros $5$ días.
  \subtask{70} Se probarán varios casos de prueba sin restricciones adicionales.
\end{scoreDescription}

\begin{sampleDescription}
\sampleIO{sample-3}
\vspace{-1em}
\textbf{Explicación:} En este caso coinciden el primer día y por lo tanto deben esperar 0 días.
\\

\sampleIO{sample-1}
\vspace{-1em}
\textbf{Explicación:} Las siguientes tablas muestran los calendarios
de Ricardo y su nóctulo a partir de los cuales podemos concluir que deben esperar 8 días
para que ambos trabajen al mismo tiempo.
\begin{table}[h]
\hspace{1em}
\begin{tabular}{|l|l|l|l|l|l|l|}
\hline
Lunes      & Martes                             & Miércoles                          & Jueves   & Viernes  & Sábado     & Domingo    \\ \hline
00 - Noche & 01 - Noche                         & 02 - Noche                         & 03 - Día & 04 - Día & 05 - Noche & 06 - Noche \\ \hline
07 - Noche & \textbf{08 - Noche} & 09 - Noche & 10 - Día & 11 - Día & 12 - Noche & 13 - Noche \\ \hline
\end{tabular}
\end{table}

\begin{table}[h]
\hspace{1em}
\begin{tabular}{|l|l|l|l|l|}
\hline
Noctulunes & Noctumartes & Noctumiércoles & Noctujueves         & Noctuviernes \\ \hline
00 - Día   & 01 - Día    & 02 - Día       & 03 - Noche          & 04 - Noche   \\ \hline
05 - Día   & 06 - Día    & 07 - Día       & \textbf{08 - Noche} & 09 - Noche   \\ \hline
\end{tabular}
\end{table}

\vspace{1em}
\sampleIO{sample-2}
\vspace{-1em}
\textbf{Explicación:} Dado que Ricardo siempre trabaja en la noche y su nóctulo siempre
trabaja en el día, sus horarios libres nunca coinciden.
\end{sampleDescription}

\end{document}
