\documentclass{oci}
\usepackage[utf8]{inputenc}
\usepackage{lipsum}


\title{Baldosas}

\begin{document}
\begin{problemDescription}
  Aburrida con la monótona rutina del día a día, Ana finalmente quiso darle un quiebre a esta, por lo que decidió unirse
  a la Organización de Cocinas Impecables (OCI). Para poder postular, su cocina debe cumplir ciertos requisitos. Uno de ellos
  exige que el piso de su cocina tenga un novedoso patrón de baldosas con estilo de ajedrez.

  Este patrón consiste en formar una grilla de baldosas donde se alternan las de color blanco con las de color negro. Además,
  la nueva moda indica que el cuadrado superior izquierdo siempre debe ser de color blanco.

  La cocina de Ana es un rectángulo de $n \times m$ metros donde $n$ es la altura, $m$ es el ancho, y además cada baldosa es un cuadrado de $1 \times 1$ metros.

  \begin{center}
    \includegraphics[width=0.25\textwidth]{example-checkerboard.pdf} \\
    Una cocina de $3 \times 4$ metros.
  \end{center}

  Como Ana no desea desperdiciar material, necesita comprar la cantidad exacta de baldosas que utilizará, por lo que necesita
  de un programa que pueda calcularlo por ella. La tienda cierra en menos de $4$ horas, por lo que necesita que le hagas un
  programa que permita calcularlo lo más rápido posible.
\end{problemDescription}

\begin{inputDescription}
  La entrada consiste en una sola línea con dos enteros $n$ y $m$ ($1 \leq n, m \leq 10^9$), representando el largo y
  ancho de la cocina respectivamente.
\end{inputDescription}

\begin{outputDescription}
  La salida debe contener una línea con dos enteros $x$ e $y$, representando la cantidad de baldosas blancas y negras que
  Ana debe comprar respectivamente.
\end{outputDescription}

\clearpage
\begin{scoreDescription}
  \subtask{25} Se probarán varios casos de prueba donde $1 \leq n, m \leq 10^3$.
  \subtask{25} Se probarán varios casos de prueba donde $n = 1$ y $1 \leq m \leq 10^9$.
  \subtask{25} Se probarán varios casos de prueba donde $1 \leq n, m \leq 10^9$ y además ambos son pares.
  \subtask{25} Se probarán varios casos de prueba sin restricciones adicionales.
\end{scoreDescription}

\begin{sampleDescription}
\sampleIO{sample-1}
\sampleIO{sample-2}
\end{sampleDescription}

\end{document}
