\documentclass{oci}
\usepackage[utf8]{inputenc}
\usepackage{lipsum}


\title{Baldosas}

\begin{document}
\begin{problemDescription}
  Después de mucho meditarlo, Ana ha decidido
  postular a la Organización de Cocinas Impecables (OCI).
  %
  La OCI es muy estricta y pone ciertos requisitos que
  las cocinas de todos sus miembros deben cumplir.
  %
  Uno de ellos exige que el piso de la cocina tenga un novedoso
  patrón de baldosas con estilo de ajedrez.
  %
  Este patrón consiste en formar una grilla
  donde se alternan baldosas de color blanco y color negro.
  %
  Adicionalmente, de acuerdo a la última moda, el cuadrado
  superior izquierdo siempre debe ser de color blanco.

  La cocina de Ana no cumple con el requisito y por lo tanto
  debe remodelarla.
  %
  Su cocina es un rectángulo de $n$ metros de alto y $m$
  metros de ancho.
  %
  Para poder formar el patrón, Ana ha decidido comprar baldosas de
  $1\times 1$ metros y por lo tanto necesitará un total de $n\times m$
  baldosas para cubrir todo su piso.

  \begin{center}
    \includegraphics[width=0.25\textwidth]{example-checkerboard.pdf} \\
    Una cocina de $3 \times 4$ metros.
  \end{center}

  Ana no quiere desperdiciar material, y por lo tanto desea comprar
  la cantidad exacta de baldosas que requiere de cada color.
  %
  La tienda cierra en menos de $4$ horas, y no hay tiempo que perder.
  %
  ?`Podrías ayudarla escribiendo un programa que determine la cantidad
  exacta de baldosas necesarias de cada color?
\end{problemDescription}

\begin{inputDescription}
  La entrada consiste en una sola línea con dos enteros $n$ y $m$
  ($1 \leq n \cdot m \leq 10^{18}$), representando respectivamente
  el alto y ancho de la cocina.
\end{inputDescription}

\begin{outputDescription}
  La salida debe contener una única línea con dos enteros $x$ e $y$,
  representando respectivamente la cantidad de baldosas blancas y
  la cantidad de baldosas negras que Ana debe comprar para poder
  completar el patrón.
\end{outputDescription}

\clearpage
\begin{scoreDescription}
  \subtask{25} Se probarán varios casos de prueba donde $1\leq n \leq 10^3$ y $1\leq m\leq 10^3$.
  \subtask{25} Se probarán varios casos de prueba donde $n = 1$ y $1 \leq m \leq 10^{18}$.
  \subtask{25} Se probarán varios casos de prueba donde $1 \leq n \cdot m \leq 10^{18}$
  y además ambos son pares.
  \subtask{25} Se probarán varios casos de prueba sin restricciones adicionales.
\end{scoreDescription}

\begin{sampleDescription}
\sampleIO{sample-1}
\sampleIO{sample-2}
\end{sampleDescription}

\end{document}
